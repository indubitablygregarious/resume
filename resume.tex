% TEX program = xelatex
%%%%%%%%%%%%%%%%%%%%%%%%%%%%%%%%%%%%%%%%%
% Important note:
% This template must be compiled with XeLaTeX, the below lines will ensure this
%!TEX TS-program = xelatex
%!TEX encoding = UTF-8 Unicode
%
%%%%%%%%%%%%%%%%%%%%%%%%%%%%%%%%%%%%%%%%%

%----------------------------------------------------------------------------------------
%	PACKAGES AND OTHER DOCUMENT CONFIGURATIONS
%----------------------------------------------------------------------------------------

\documentclass[11pt, a4paper]{resume}
\usepackage{graphicx}
\usepackage{fancyhdr}
\usepackage{graphicx}
\usepackage{tikz}
\usepackage{eso-pic}
\usepackage{tcolorbox}
\usepackage{hyperref}

% Configure page margins with geometry
\setlength{\sidemargin}{1.2cm}
\geometry{left=\sidemargin, top=1.5cm, right=\sidemargin, bottom=2cm, footskip=.5cm}

\fontdir[fonts/] % Specify the location of the included fonts

% Color for highlights
\definecolor{awesome}{HTML}{18678d}

% Colors for text - uncomment and modify
%\definecolor{darktext}{HTML}{414141}
%\definecolor{text}{HTML}{414141}
%\definecolor{graytext}{HTML}{414141}
%\definecolor{lighttext}{HTML}{414141}

\renewcommand{\headerSep}{\quad\textbar\quad} % If you would like to change the social information separator from a pipe (|) to something else

%----------------------------------------------------------------------------------------
%          Header Information
%----------------------------------------------------------------------------------------

\name{Pete}{Lesko}
\address{Columbia, MD}
\mobile{(+1) 555 NOT FAKE}

\email{your.email@gmail.com}
\github{indubitablygregarious}
\linkedin{pete-lesko-5109904}

%----------------------------------------------------------------------------------------
%          PHOTO
%          There are a lot of reasons not to include your photo
%          Please reference the below article as to why
%----------------------------------------------------------------------------------------
%  https://www.indeed.com/career-advice/resumes-cover-letters/should-i-put-my-picture-on-my-resume 
%\photo[circle,edge,right]{mypic.png}

\position{Platform Engineering Manager}
%\quote{``Make the change that you want to see in the world."} % A quote or statement

%----------------------------------------------------------------------------------------
%\rhead{\includegraphics[width=3cm]{pete.png}} % adjust the width as needed
%\rhead{\raisebox{-102px}
%    {\begin{tikzpicture}%
%        \node[circle, draw=darkgray, line width=0.3mm, inner sep=1.2cm, fill overzoom image=pete.png] () {};
%    \end{tikzpicture}}
%} % adjust the width as needed
% a nice blue stripe on the left of the page
\AddToShipoutPictureBG{%
\begin{tikzpicture}[overlay,remember picture]
\shade[left color=white, right color=awesome] (current page.north west) rectangle ([xshift=2cm]current page.south west);
\end{tikzpicture}
}
\begin{document}

\makecvheader % Print the header
\makecvfooter{\today}{Pete Lesko~~~·~~~Resume}{\thepage} % Specify the letter footer with 3 arguments: (<left>, <center>, <right>), leave any of these blank if they are not needed


\begin{cvsummary}
    \begin{tcolorbox}[colback=white, rounded corners, boxrule=0pt, colframe=white]
    %Software engineer with close to 4 years experience, mainly coding in Python. I care about code quality and writing readable, explicit code. I am currently experimenting with Rust on my free time to expand my skillset. I am looking for a role where I can be challenged and learn from industry experts.
    Lorem ipsum dolor sit amet, consectetur adipiscing elit, sed do eiusmod tempor incididunt ut labore et dolore magna aliqua. Ut enim ad minim veniam, quis nostrud exercitation ullamco laboris nisi ut aliquip ex ea commodo consequat. Duis aute irure dolor in reprehenderit in voluptate velit esse cillum.
    \end{tcolorbox}
\end{cvsummary}
\vspace{5mm}

%\begin{center}
    %%\hline{\linewidth}
%\end{center}

%----------------------------------------------------------------------------------------
%	CV/RESUME CONTENT
%	Each section is imported separately, open each file in turn to modify content
%----------------------------------------------------------------------------------------

\setlength{\headersmargin}{-0.2cm}
\setlength{\leftcolumnwidth}{5cm}
\setlength{\columnsmargin}{0.5\sidemargin}
\setlength{\timelinemargin}{0.5cm}

\begin{textblock*}{\leftcolumnwidth}(0pt, \headersmargin)
    \vspace{-2mm}
    %----------------------------------------------------------------------------------------
%	CONTACT
%----------------------------------------------------------------------------------------
\cvsidesection{Contact}

\begin{tcolorbox}[colback=white, rounded corners, boxrule=0pt, colframe=white, top=0pt, bottom=0pt]
\begin{tabular*}{\leftcolumnwidth - 2\sidemargin + 2\iconswidth}{c@{}C{0.45\leftcolumnwidth - 0.5\sidemargin + 0.5\iconswidth}@{}c@{}R{0.55\leftcolumnwidth - 0.50\sidemargin + \iconswidth}}
    \caddressicon & \multicolumn{3}{r}{\caddress}\\
    \cmobileicon & \multicolumn{3}{r}{\cmobile}\\
    \cemailicon & \multicolumn{3}{r}{\cemail}\\
    \cgiticon & \multicolumn{3}{r}{\cgituser}\\
    \clinkedinicon & \multicolumn{3}{r}{\clinkedinuser}\\
\end{tabular*}
\end{tcolorbox}

    \vspace{1mm}
    %----------------------------------------------------------------------------------------
%	SECTION TITLE
%----------------------------------------------------------------------------------------

\cvsidesection{Skills}

\begin{cvskills}

% ------------------------------------------------
\skillcategory{People Skills}

\begin{skillstable}
  \skill{Leadership}\skill{Empathy}\skill{Hiring}\\[3pt]
  \skillbig{Tech Roadmapping}\skill{Mentorship}\\[3pt]
  \skillbig{Product Management}\skill{Agile}
\end{skillstable}

% ------------------------------------------------
\skillcategory{Ops Stack}

\begin{skillstable}
  \skill{Kubernetes}\skill{Terraform}\skill{Helm}\\[3pt]
  \skill{GitLab}\skill{GitHub}\skill{Docker}
\end{skillstable}

% ------------------------------------------------
\skillcategory{Databases / Services}

\begin{skillstable}
    \skill{PostgreSQL}\skill{DataDog}\skill{Sumologic}\\[3pt]
    \skill{AWS}\skill{EKS}\skill{RDS}\\[3pt]
    \skill{EC2}\skillbig{Cost Explorer}
\end{skillstable}

% ------------------------------------------------
\skillcategory{Programming Languages}

\begin{skillstable}
  \skill{Python}\skill{Bash}\skill{JavaScript}\\[3pt]
  \skill{C}\skill{LaTeX}\skill{Perl}
\end{skillstable}

% ------------------------------------------------
\skillcategory{Software}

\begin{skillstable}
    \skillbig{Generative AI}\skill{Jira}\\[3pt]
    \skillbig{Google Suite}\skill{Orca}\\[3pt]
    \skill{Automation}\skill{Monitoring}\skill{Alerting}
\end{skillstable}

% ------------------------------------------------
\skillcategory{Languages}

\vspace{-1mm}
\begin{tcolorbox}[colback=white, rounded corners=north, rounded corners=south, boxrule=0pt, colframe=white, top=6pt, bottom=6pt]
  \begin{skillstable}
      \languagestyle{English \textit{(native)}, French \textit{(pour la plupart intelligible)}}
  \end{skillstable}
\end{tcolorbox}

\end{cvskills}

\vspace{2mm}

\end{textblock*}

\begin{textblock*}
    {\paperwidth - 2\sidemargin - \leftcolumnwidth - \columnsmargin - 2\timelinemargin}
    (\leftcolumnwidth + \columnsmargin + 2\timelinemargin, \headersmargin)
    \vspace{-2mm}
    \input{sections/experience.tex}
\end{textblock*}

\timeline{C}{C} % draw the timeline starting from section D to section D

%----------------------------------------------------------------------------------------
% Debug - Show lengths used in the document
%\newcommand{\pos}[2]{([xshift=#1,yshift=-#2]current page.north west)}
%\begin{tikzpicture}[
    %remember picture,
    %overlay,
    %arrow/.style={stealth-stealth},
%]
    %\draw[arrow, thick] \pos{0pt}{6cm} -- \pos{\sidemargin}{6cm};
    %\draw[arrow, thick] \pos{\leftcolumnwidth + \sidemargin}{6cm} -- \pos{\leftcolumnwidth + \sidemargin + \sidemargin}{6cm};
    %\draw[arrow, thick] \pos{\leftcolumnwidth + \sidemargin}{6cm} -- \pos{\leftcolumnwidth + \sidemargin + \sidemargin}{6cm};
%\end{tikzpicture}

\newpage


\begin{textblock*}{\leftcolumnwidth}(0pt, \headersmargin)
    \vspace{-2mm}
    %----------------------------------------------------------------------------------------
%	HOBBIES
%----------------------------------------------------------------------------------------

\cvsidesection{Hobbies}

\begin{tcolorbox}[colback=white, rounded corners=north, rounded corners=south, boxrule=0pt, colframe=white, top=0pt, bottom=0pt]
\begin{tabular*}{\leftcolumnwidth - \sidemargin}{cR{\leftcolumnwidth - \sidemargin + \iconswidth}@{}}
    \hobbyiconstyle{\icon{󱗖}} & \hobbiesstyle{Ducimus qui blanditiis} \\
    \hobbyiconstyle{\icon{}} & \hobbiesstyle{Quidem rerum facilis est et} \\
    \hobbyiconstyle{\icon{}} & \hobbiesstyle{Minus id quod maxime placeat} \\
    \hobbyiconstyle{\icon{󰇦}} & \hobbiesstyle{Temporibus autem quibusdam} \\
\end{tabular*}
\end{tcolorbox}

\end{textblock*}

% TODO this kills the dots but having a header on both pages would be nice
%\makecvheader % Print the header

\begin{textblock*}
    {\paperwidth - 2\sidemargin - \leftcolumnwidth - \columnsmargin - 2\timelinemargin}
    (\leftcolumnwidth + \columnsmargin + 2\timelinemargin, \headersmargin)
    %----------------------------------------------------------------------------------------
%	SECTION TITLE
%----------------------------------------------------------------------------------------

\cvsection{Experience continued}


%----------------------------------------------------------------------------------------
%	SECTION CONTENT
%----------------------------------------------------------------------------------------

%\cvsubsection{Aledade}

\begin{cventries}

%------------------------------------------------

\cventry
{Software Engineer 3} % Job title
{Good company} % Organization
{Columbia, MD} % Location
{Nov 2013 - Sep 2015\duration{2 years}} % Date(s)
{\begin{cvitems}
\item {Filler}
\item {Filler}
\item {Filler}
\item {Filler}
\end{cvitems}
}

%------------------------------------------------

\cventry
{Software Engineer 3} % Job title
{Good company} % Organization
{Columbia, MD} % Location
{Feb 2011 - Nov 2013\duration{2.5 years}} % Date(s)
{\begin{cvitems}
\item {Filler}
\item {Filler}
\item {Filler}
\item {Filler}
\end{cvitems}
}




\end{cventries}


    \input{sections/education.tex}
    \input{sections/projects.tex}
\end{textblock*}

\timeline{E}{G} % draw the timeline starting from section E to section E

\end{document}
